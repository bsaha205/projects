\documentclass[]{report}
\usepackage{relsize}
\usepackage{listings}
\usepackage{textcomp}
\usepackage{hyperref}
\usepackage{wasysym}
\usepackage{verbatim}
%\lstset{language=Pascal}

\iffalse
\lstset{%
  basicstyle=\small\ttfamily,
  language=[LaTeX]{TeX},
  backgroundcolor=\color{white},
  numbers=left, numberstyle=\tiny, stepnumber=1, numbersep=5pt,
    commentstyle=\color{red},
    showstringspaces=false,
    keywordstyle=\color{blue}\bfseries,
    morekeywords={align,begin},
    pos=l,
}
\fi

\title{How to Write\\
Research Papers in Computer Science}
\author{Bishwajit Saha \\
		Student ID: 1205043\\
		Dept: CSE \\
		Sec: A
		}
\date{\today}

\begin{document}
  \maketitle
  \begin{abstract}
  This report tells all the process for writing a nice and worthy thesis paper,research paper or report.For publishing research papers in a well-known scientifiq magazines,that research papers must be well written.And this report says exactly that by using latex environment.
  \end{abstract}
  \tableofcontents
  
  
  \chapter{Research in Computer Science and Engineering}
  \paragraph{Steps to follow:}
  \begin{itemize}
  \item Study and explore your area of interest.
computer science and engineering;
  \item Choose a research problem.
  \item Find one or two co-researchers and form a research group.
 \item Read related research papers published in good journals
and conferences and present those papers in the group, by
rotation.
 \item Sit frequently for brainstorming on the problem and try to
find non-trivial results.
\item Find good results around the problem and write papers.
  \end{itemize}
  
  
  \chapter{Issues of Technical Writing}
  
  \section{Why do we need writing?}
  \begin{enumerate}
  \item To express our ideas.
  \item Writing good mathematical explanations will improve your
knowledge and understanding of the mathematical ideas you
encounter.
\item Putting an idea on paper requires careful thought and attention.
Hence, mathematics which is written clearly and carefully is
more likely to be correct.
\item Specifically ....
\begin{itemize}
\item For publishing your research results as research papers.
\item For publishing your study and research findings as theses
or dissertation to fulfil your degree requirements.
\item Writing a technical text books for students.
\item Writing review reports.
\item Technical study report.
\item Etc.
\end{itemize}
  \end{enumerate}
  
  \section{Who is Your Audience?}
   \begin{itemize}
   \item Your Peers!!
   \item Think someone to whom you could be explaining your ideas.
   \item You should consider
   \begin{itemize}
   \item What questions might be asked.
   \item What confusions might arise.
   \item which details you might need to trot out and explain.
   \end{itemize}
  \end{itemize}
  
  
  \section{Aspects of Technical Writing}
  \subsection{}
  \paragraph{As a valued customer of XYZ company, your call is very
important to us.}
\begin{center}
\emph{{\large What is the wrong with this sentence?}}
\end{center}
\begin{verse}
{\large Here:} "Your call" is a valued  customer.
\end{verse}
\begin{verse}
{\large Good:} You are a valued customer of XYZ company, and your call is
very important to us.
\end{verse}
\begin{center}
\textbf{Or}
\end{center}
\begin{verse}
{\large Good:} Because you are a valued customer of XYZ company, your call
is important to us.
\end{verse}
\begin{center}
\textbf {{\large Say What You Mean; \\
	    Mean What You Say.} }
\end{center}
\subsection{}
\paragraph{The conjecture of Gause (1930) is false.}
\paragraph{The lemmas of Euler (1766) and the example of Abel (1827)
led Gause to conjecture that all semistable curves are modular.
The conjecture was widely beleived, and more than fifty papers
were written by Jacobi, Dirichlet, and Galios in support of it. To
everyones’s surprise and dismay, a counter example was
produced by Frobenius in 1902. This counterexample opened
many doors.}
\paragraph{}
\begin{center}
\textbf{{\large Must make choice to convey most effectively a given message
and the sprit of the message.}}
\end{center}

\subsection{}
\paragraph{Let X be a compact metric subspace of the space Y. If f is
continuous, R-valued function on the space then it assumes
both a maximum and a minimum value.}
\begin{center}
\emph{{\large Here: X and Y are defined, but not used.}}
\end{center}
\begin{verse}
{\large Good:} Let X be a compact metric space. If f is a continuous,
real-valued function on X then f assumes both a maximum and
a minimum value. 
\end{verse}

\begin{verse}
{\large Better:} A continuous, real valued function on a compact metric
space assumes both a maximum and a minimum value.
\end{verse}
\begin{center}
\textbf{{\large Minimize the number of notations.}}
\end{center}

\subsection{}
$\forall x \exists y,x \geq 0 \Rightarrow y^2=x$
\begin{center}
\emph{{\large difficult to read!}}
\end{center}
\begin{verse}
{\large Better:} Every nonnegative real number has a squre root.
\end{verse}
\begin{center}
\textbf{{\large Minimize the number of notations.}}
\end{center}

\subsection{}
\paragraph{All continuous functions have a maximum.}
\begin{center}
\emph{{\large All continuous functions share the same maximum!!}}
\end{center}
\begin{verse}
{\large Better:} Every continuous function has a maximum.
\end{verse}
\begin{verse}
{\large More preciously:} Each continuous function has a maximum.
\end{verse}
\begin{center}
\textbf{{\large Be precise, avoid ambiguity.}}
\end{center}

  \chapter{General Guidelines for Technical Writing}
  \section{Rules and Practices of Writing}
  \begin{description}
  \item[**   ] Be careful about the language: Grammar, sentence
formations, spellings, punctuation etc.
\item[**   ] Each paragraph should represent a specific idea.
\item[**  ] Smooth transition from
\begin{itemize}
\item One paragraph to the next
\item One sentence to the next
\end{itemize}
\item[**  ] Write short and simple sentences.
\item[**  ] The opening paragraph of a section should be the best
paragraph of the section.
\item[**  ] The opening sentence of a paragraph should be the best
sentence of the paragraph.
\item[**  ] Every statement should be precise and correct.
  
  \paragraph{}
 \textbf {\large {Example}} 
 \begin{verse}
 "The problem stated above is difficult"
 \end{verse}
  \begin{itemize}
  \item Difficult for whom?
  \item NP-complete?
  \item Believed by you?
  \item Believed by others?
  \item Proved by
someone?
  \end{itemize}
  \item[**  ] Statement should be logical. Avoid sentence of the form
"An x is y."
\begin{verse}
{\large Bad:} An important method for internal sorting is
quicksort.
\end{verse}
\begin{verse}
{\large Good:} Quicksort is an important method for internal
sorting, because ...
\end{verse}
\item[**  ] Vary the sentence structure and the choice of words to
avoid monotony. But use parallelism when parallel
concepts are being discussed.
\begin{verse}
{\large Bad:} Formerly, science was taught by the textbook
method, while now the laboratory method is
employed.
\end{verse}
\begin{verse}
{\large Good:} Formerly, science was taught by the textbook
method; now it is taught by the laboratory
method.
\end{verse}

\item[**  ] Do not omit "that" when it helps the reader to parse
sentence
\begin{verse}
{\large Bad:} Assume G is a graph.
\end{verse}
\begin{verse}
{\large Good:} Assume that G is a graph.
\end{verse}
  
  \item[**  ] There is a definite rhythm in sentences. Read what you
have written, and change the wording if it does not flow
smoothly.
\paragraph{}
\textbf{\large Active or Passive:}
In computer science writing active voice is
preferred.
\begin{verse}
{\large Bad:} The following result can now be
proved.
\end{verse}
\begin{verse}
{\large Good:}We can now prove the following
theorem.
\end{verse}
\begin{center}
\textbf{{\large I or We Always use "we" even you are a single author.}}
\end{center}
\end{description}

\section{Important Points for Mathematical Writing}
\subsection{Separating symbols in formulas:}
\begin{verse}
{\large Bad:} Consider \begin{math}
{S_q}
\end{math} , $q < p.$
\end{verse}
\begin{verse}
{\large Good:} Consider \begin{math}
{S_q}
\end{math} ,where $q < p.$
\end{verse}

\subsection{Not starting sentence with a symbol:}
\begin{verse}
{\large Bad:} G has n vertices.
\end{verse}
\begin{verse}
{\large Good:} The graph G has n vertices.
\end{verse}

\subsection{Not using symbols $ \forall, \exists ,\in : $}
\begin{verse}
replace them by corresponding
words.
\end{verse}
\subsection{Defining symbols before use: }
\begin{verse}
{\large Bad:} Algorithm XYZ finds a drawing of G in
O(n + m) time, where n and m are the
numbers of vertices and edges, respectively.
\end{verse}
\begin{verse}
{\large Good:} Let G be a graph of n vertices and m edges.
Then Algorithm XYZ finds a drawing of G in
O(n + m) time.
\end{verse}

\subsection{Not using quotations in mathematics papers frequently:}
\begin{verse}
{\large Bad:} As Methuselah used to say, " When the going
gets tough, the tough get going".
\end{verse}
\begin{verse}
{\large Good:} As Methuselah used to say, " When the going
gets tough, the tough get going."
\end{verse}

\subsection{Placing punctuation marks rightly:}
\begin{verse}
Commas and periods should be placed inside quotation
marks, and colons and semicolons outside quotation
marks
\end{verse}

\subsection{Completing Sentence rightly:}
\begin{verse}
{\large Bad:} We now have the following
\textbf{Theorem}. \emph {H(x)} is continuous.
\end{verse}
\begin{verse}
{\large Good:} We can now prove the following result.
\textbf {Theorem}. The function \emph {H(x)} defined in (5) is
continuous.
\end{verse}

\subsection{Being self-contained: }
\begin{verse}
The statement of a theorem should usually be
self-contained, not depending on the assumptions on the
previous text.
\end {verse}

\subsection{Fact, Lemma, Theorem, Corollary:}
\paragraph{}
\textbf {Fact :}
\begin{verse}
A proposition which is obviously true. Usually
does not need a proof.
\end{verse}
\paragraph{}
\textbf {Lemma :}
\begin{verse}
A proposition which will be used to prove other
propositions. A proof is needed.
\end{verse}
\paragraph{}
\textbf {Theorem :}
\begin{verse}
A proposition which gives a main result of the
paper. A proof is needed.
\end{verse}
\paragraph{}
\textbf {Corollary :}
\begin{verse}
Immediate from a theorem or a lemma.
\end{verse}

\subsection{Capitalizing names of theorems,lemmas..:}
\begin{verse}
{\large Wrong:} By lemma 3, we have ...
\end{verse}
\begin{verse}
{\large Correct:}By Lemma 3, we have ...
\end{verse}
\paragraph{}
\begin{verse}
{\large Wrong:} We now have the following Lemma.
\end{verse}
\begin{verse}
{\large Correct:}We now have the following lemma.
\end{verse}

\paragraph{}
\begin{verse}
{\large Wrong:} A maximal matching is illustrated in figure 5(a).
\end{verse}
\begin{verse}
{\large Correct:}A maximal matching is illustrated in Figure 5(a).
\end{verse}

\paragraph{}
\begin{verse}
{\large Wrong:} In section 3 we deal with orthogonal drawings of
planar graphs.
\end{verse}
\begin{verse}
{\large Correct:}In Section 3 we deal with orthogonal drawings of
planar graphs.
\end{verse}

\subsection{Spelling numbers or not: }
\begin{verse}
{\large Wrong:} There are 5 vertices on the outer face.
\end{verse}
\begin{verse}
{\large Correct:}There are five vertices on the outer face.
\end{verse}
\paragraph{}
\begin{verse}
{\large Wrong:} The count was increased by two.
\end{verse}
\begin{verse}
{\large Correct:}The count was increased by 2.
\end{verse}
\paragraph{}
\begin{verse}
{\large Wrong:} The graph has eighty embeddings.
\end{verse}
\begin{verse}
{\large Correct:}The graph has 80 embeddings.
\end{verse}

\subsection{Displaying important formulas:}
\begin{verse}
Display important formulas on a line by themselves. If you
need to refer to some of these formulas from remote parts
of the text, give reference numbers to all of the most
important ones, even if they are not referenced.
\end{verse}
\section{Some more points...}
\begin{itemize}
\item Do not overuse commas.
\begin{verse}
{\large Bad:}I went to the store, to buy some potatoes.
\end{verse}
\begin{verse}
{\large Bad:}In this paper, we give a linear-time algorithm.
\end{verse}
\item Do not use contraction in formal writing. \textbf{Bad:} don't, I'm ...
\end{itemize}

\chapter{Acknowledgement}
\begin{itemize}
\item \textbf{Sources:}
\begin{itemize}
\item D. E. Knuth, T. Larrabee and P. M. Robers, Mathematical
Writing, MAA Notes, 14, The Mathematical Association of
America, 1989.
\item S. G. Krantz, A primer of Mathematical Writing, American
Mathematical Society, 1997.
\item R. Andonie and I. Dzitac, How to write a good paper in
computer science and how will it be measured by ISI web of
knowledge, Int. J. of Computers, Communications and
Control, 4, pp. 432-446, 2010.
\item U. Khedker, How to Write a Good Paper? Indian Institute of
Technology, Bombay (slides).
\item https://cs.uwaterloo.ca/ brecht/thesis-hints.html, accessed
on August 29, 2013.
\end{itemize}
\end{itemize}
\end{document}
