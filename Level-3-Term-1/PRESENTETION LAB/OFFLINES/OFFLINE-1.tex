\documentclass[]{report}
\usepackage{relsize}
\usepackage{listings}
\usepackage{textcomp}
\usepackage{hyperref}
\usepackage{wasysym}
\usepackage{verbatim}
%\lstset{language=Pascal}

\iffalse
\lstset{%
  basicstyle=\small\ttfamily,
  language=[LaTeX]{TeX},
  backgroundcolor=\color{white},
  numbers=left, numberstyle=\tiny, stepnumber=1, numbersep=5pt,
    commentstyle=\color{red},
    showstringspaces=false,
    keywordstyle=\color{blue}\bfseries,
    morekeywords={align,begin},
    pos=l,
}
\fi

\title{CSE 300\\
Technical Writing and Presentation}
\author{Bishwajit Saha \\
		Student ID: 1205043\\
		Dept: CSE \\
		Sec: A
		}
\date{\today}

\begin{document}
  \maketitle
  \begin{abstract}
  This is abstract
  \end{abstract}
  \tableofcontents
  
  
  \chapter{Technical Writing and 						Presentation}
  {\Huge Syllabus}
  \paragraph{}
  \begin{itemize}
  \item Issues of technical writing and oral presentation in
computer science and engineering;
  \item Writing Styles of definitions, propositions, theorem and
proofs;
  \item Preparation of reports, research papers, theses and books:
abstract, preface, content, bibliography and index
 \item Writing book reviews and referee reports;
 \item Writing tools: LaTex; Diagram Drawing software;
presentation tools.
  \end{itemize}
  
  
  \chapter{Issues of Technical Writing}
  
  \section{Why do we need writing?}
  \begin{enumerate}
  \item To express our ideas.
  \item Writing good mathematical explanations will improve your
knowledge and understanding of the mathematical ideas you
encounter.
\item Putting an idea on paper requires careful thought and attention.
Hence, mathematics which is written clearly and carefully is
more likely to be correct.
\item Specifically ....
\begin{itemize}
\item For publishing your research results as research papers.
\item For publishing your study and research findings as theses
or dissertation to fulfil your degree requirements.
\item Writing a technical text books for students.
\item Writing review reports.
\item Technical study report.
\item Etc.
\end{itemize}
  \end{enumerate}
  
  \section{Who is Your Audience?}
   \begin{itemize}
   \item Your Peers!!
   \item Think someone to whom you could be explaining your ideas.
   \item You should consider
   \begin{itemize}
   \item What questions might be asked.
   \item What confusions might arise.
   \item which details you might need to trot out and explain.
   \end{itemize}
  \end{itemize}
  
  
  \section{Aspects of Technical Writing}
  \paragraph{As a valued customer of XYZ company, your call is very
important to us.}
\begin{center}
\emph{{\Large What is the wrong with this sentence?}}
\end{center}
  \chapter{First Appendix}
  The \verb|\appendix| macro can be used to indicate that following sections (for articles) or chapters (for books) are to be numbered as appendices.
  
  \chapter{Last Note}
\end{document}
