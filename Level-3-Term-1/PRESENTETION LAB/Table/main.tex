\documentclass{article}
\usepackage[utf8]{inputenc}
\usepackage{hyperref}
\usepackage{booktabs}
\usepackage{array}
\usepackage{textcomp}
\usepackage{multirow}
\usepackage{amsmath}
\usepackage{xcolor}

\renewcommand{\arraystretch}{1.3} % extending the space between rows, default 1.0

\title{Table Example}
\author{Tanvir Ahmed Khan}
\date{\today}

\begin{document}

\maketitle

\section{Introduction}
%
This \href{http://en.wikibooks.org/wiki/LaTeX/Tables}{article} on \LaTeX\ tables from wikibooks community is the most comprehensive guide we have seen so far on how to construct tables in \LaTeX. Another good introductory article can be found \href{https://www.sharelatex.com/learn/Tables}{here}. However, they all taught you how to construct tables in \LaTeX, what they are lacking is how to present nice tables for technical publication and visual presentation. This \href{http://www.inf.ethz.ch/personal/markusp/teaching/guides/guide-tables.pdf}{document} shows some overview of that.

We will see some rudimentary table examples in this article. Most of them are taken from this \href{http://dragonbook.stanford.edu/}{book}, some of them are taken from the sources listed above. Feel free to explore them for further reading.

\section{Simple Tables}
%
Let us start with a simple table example. Table~\ref{table:sddcalculator} illustrates a basic example of \LaTeX\ table.
\begin{table}[!h] % table environment to add caption and label, [position_specifier=h (here), t (top of the page), b (bottom of the page), p (dedicated page), !(override default float restrictions)]
\begin{center} % centering the table and caption
\begin{tabular}{lll} % table declaration, number of columns using {table spec}, l (left-justified column), c (centered), r (right-justified), | (vertical line between columns)
\toprule % top horizontal line with extra width requires booktabs package
 & PRODUCTION & SEMANTIC RULES\\ % use & to separate columns and use \\ to end row
\hline % horizontal line after the header row
1) & $L \rightarrow E \textbf{n}$ & $L.val=E.val$\\
2) & $E \rightarrow E_{1}\ +\ T$ & $E.val=E_{1}.val+T.val$\\
3) & $E \rightarrow T$ & $E.val=T.val$\\
4) & $T \rightarrow T_{1}\ *\ F$ & $T.val=T_{1}.val\times F.val$\\
5) & $T \rightarrow F$ & $T.val=F.val$\\
6) & $F \rightarrow (E)$ & $F.val=E.val$\\
7) & $F \rightarrow \textbf{digit}$ & $F.val=\textbf{digit}$.lexval\\
\bottomrule
\end{tabular}
\caption{Syntax-directed definition of a simple desk calculator}
\label{table:sddcalculator}
\end{center}
\end{table}


Now let us look at another table example where we have to wrap texts in table cells. \LaTeX\ does not do that automatically.
\begin{table}[!h]
\begin{center}
\begin{tabular}{cc}
PRODUCTION & SEMANTIC RULES\\
$A \rightarrow B$ & \shortstack{$A.s=B.i;$\\$B.i=A.s+1$}\\
\end{tabular}
\end{center}
\end{table}


{
\def\arraystretch{1.3}
%\large
\begin{center}
\begin{tabular}{l>{\centering\arraybackslash}p{0.08\textwidth}>{\centering\arraybackslash}p{0.08\textwidth}>{\centering\arraybackslash}p{0.08\textwidth}>{\centering\arraybackslash}p{0.15\textwidth}}
\toprule
\textbf{Instructions \textrightarrow} & \textbf{FP} & \textbf{INT} & \textbf{L/S} & \textbf{BRANCH}\\
\midrule
\textbf{Instruction Count ($\bf\times 10^6$)}& 50 & 110 & 80 & 16\\
\textbf{CPI}& 1 & 1 & 4 & 2\\
\bottomrule
\end{tabular}
\end{center}
}


\begin{center}
\begin{tabular}{|p{0.6\textwidth}|p{0.4\textwidth}|}
\hline
\multirow{3}{*}{Execution time after improvement} & =  \\
 & = \\
& =  {128} ($ms$)\\
\hline
\multirow{2}{*}{Execution time{ without FP}} & = \\
 & = {231}  ($ms$)\\
\hline
\end{tabular}
\end{center}



\newcommand{\zero}{\textcolor{red}{0}}
\newcommand{\one}{\textcolor{blue}{1}}
{
\scriptsize
\begin{center}
\begin{tabular}{|c|c|c|}
\hline
\textbf{Control Signal} & \zero & \one\\
\hline
RegDst & Write register address = rt & Write register address = rd \\
\hline
RegWrite & - & Write register \\
\hline
AluSrc & ALU Second Operand = Read data 2 & ALU Second Operand = lower 16-bit of instruction \\
\hline
PCSrc & PC=PC+4 & PC=branch target\\
\hline
MemRead & - & Read data from memory\\
\hline
MemWrite & - & Write data into memory\\
\hline
MemtoReg & Register Write Data from ALU & Register Write Data from data memory\\
\hline
\end{tabular}
\end{center}
}

{
\scriptsize
\begin{center}
\begin{tabular}{|>{\bfseries}c|c|c|}
\hline
%\textbf
{Control Signal} & \zero & \one\\
\hline
RegDst & Write register address = rt & Write register address = rd \\
\hline
RegWrite & - & Write register \\
\hline
AluSrc & ALU Second Operand = Read data 2 & ALU Second Operand = lower 16-bit of instruction \\
\hline
PCSrc & PC=PC+4 & PC=branch target\\
\hline
MemRead & \multicolumn{1}{c}{-} & \multicolumn{1}{c}{Read data from memory}\\
\hline
MemWrite & - & Write data into memory\\
\hline
MemtoReg & Register Write Data from ALU & Register Write Data from data memory\\
\hline
\end{tabular}
\end{center}
}

\begin{center}
\begin{tabular}{|c|c|c|c|c|}
\hline
\multicolumn{4}{|c|}{ALU Control} & \multirow{2}{*}{Operation}\\
\cline{1-4}
$S_{3}$ & $S_{2}$ & $S_{1}$ & $S_{0}$ & \\
\hline 
\zero & \zero & \zero & \zero & AND \\
\hline 
\zero & \zero & \zero & \one & OR \\
\hline 
\zero & \zero & \one & \zero & ADD \\
\hline 
\zero & \one & \one & \zero & SUBTRACT \\
\hline 
\zero & \one & \one & \one & SLT \\
\hline
\end{tabular}
\end{center}


\begin{center}
\begin{tabular}{lcc}
\toprule
 & opcode & function \\
\midrule
lw & 35 & - \\
sw & 43 & - \\
beq & 4 & - \\
add & 0 & 32 \\
sub & 0 & 34 \\
AND & 0 & 36 \\
OR & 0 & 37 \\
slt & 0 & 42 \\
j & 2 & - \\
\bottomrule
\end{tabular}
\end{center}
\end{document}
